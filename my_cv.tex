\documentclass[10pt,a4paper,sans]{moderncv}
\moderncvstyle{classic}
\moderncvcolor{black}
\usepackage[scale=0.9]{geometry}
\usepackage{multicol}
\firstname{Oleg}
\familyname{Tsarkov}
\newcommand{\myurl}[1]{\color{blue}\url{#1}\color{black}}
\email{olegtsts@gmail.com}

\begin{document}
\makecvtitle

\section{Education}
\cventry{2011 -- 2013}{Yandex School of Data Analysis}{}{}{}{}
\cventry{2009 -- 2014}{Specialist (equivalent to Master's degree) in Mathematics}{Moscow State University, Moscow, Russia}{\newline GPA -- 5.0 / 5}{}{}{}{}


\section{Work experience}
\cventry{Oct 2023 -- now}{Google LLC}{Eng Manager}{Persistent Disk}{}{
	\begin{itemize}
		\item Persistent Disk is a core Google Cloud product that provides attachable block devices (disks) for all Virtual Machines in Cloud. I am leading an infra team that has been building a new generation stack (Agents stack) for serving low latency and high throughput demand workloads. Our latency on a 4k block size workload is an impressive \textbf{400 microseconds at P50}.
		\item The team has successfully built and released Hyperdisk Balanced and Hyperdisk Extreme products on Agents stack, and has achieved several scalability milestones of 10k devices per cell and subsequently \textbf{50k devices per cell}. Achieving these scalability milestones involved establishing flows in the team to drive continuous performance investigations and protective identification of improvements in the system, such as mutex contention, low level networking issues, kernel issues, Borg issues, hardware limitations.
		\item Another significant achievement in the team is meeting our \textbf{5 nine availability SLO target} for the first time in a while in 2024. The definition of SLO is P100 (absolute maximum) latency is less than 2 seconds at least 99.999\% of time. Previously SLO was tracking 4 nines, and I had to schedule, drive and launch profound infrastructure improvements to drive it to 5 nines. One of the improvements was introducing a heuristic based cross shard failover mechanic that hasn't existed in the system. Implementation of this project alone had a 6 month timeline and was delivered on time.
		\item I've put several critical initiatives (that started before I joined the team) back on track, including Throttling features for the Control Plane layer of the system and Flow Control for the Data Plane. These features increased the resilience of the system, preventing OOM failures.
		\item I've utilized chiplet-aware scheduling in Agents layer of Persistent Disk that has led to us being able to use arcadia-milan machines that were considered unusable / highly unfavourable earlier due to performance implications. These machines are a significant fraction of fleet in Google today, therefore this update has made it easier for the Agents layer of the system to secure resources in Shared Borg.
		\item My team has built a pipeline for Persistent Disk to perform stress testing in collaboration with Cloud Networking Engprod org. The testing has reached \textbf{35M IOPS} and has helped us to identify performance bottlenecks. One year later, Cloud Networking org decided that they want to reuse resources of this setup for their needs only, as it was their resources. This had a negative impact on our ability to test intense workloads in Persistent Disk. I've navigated this complex change in collaboration with multiple teams in Persistent Disk org, and we have built an alternative stress testing flow as a result. The new flow has been successfully used to quality a new Hyperdisk ML product.
		\item I'm raising the bar of our team's production oncall rotation by performing deep performance investigations and presenting them to large audiences. I've built processes to monitor the oncall load and reduce the noise in the rotation to keep the team focused on important issues while not being overloaded.
		\item In terms of people management, I've been successfully driving and facilitating promo cases for different levels in the team, L3 $\rightarrow$ L4, L4 $\rightarrow$ L5, L5 $\rightarrow$ L6. I've ensured proper balance and fairness when proposing year-end ratings, all proposed ratings both on higher and lower end were accepted at the local org's committee level, meaning they were in alignment with TLs in the org, and made sense to engineers in my team when communicated. My approach to Hiring new talent has been rigorous, prioritizing finding the right talent for the team under pressure to close on the Hiring sooner.
	\end{itemize}
}
\cventry{May 2021 -- Oct 2023}{Google LLC}{TL}{Ads Proxybidder Team}{}{
	\begin{itemize}
		\item Maintained Shopping Bidding products knowledge transfer to the new team.
		\item Migrated some of the Shopping Bidding solutions to Proxybidder Bid stack, fully owning the migration design.
		\item TLing a sub-team of 5 people. The sub-team is responsible for monitoring, automated analytics and customer experience. Doing OKR planning for the team, including accommodating external requests. Facilitating growth of L3 folks in the team.
		\item Defining a roadmap for Bidding Factorized Model improvements for both Search and Shopping channels.
		\item Proposing automated tooling to triage customer escalations. Agreeing with management that building out this tool made sense as a proper time investment. The tool has modular and extensible design. The tool was implemented for pMax and Shopping campaigns, and there are plans to onboard more campaign types.
		\item Integrating internal Bidding data with Ads Health internal debugging UI used by gTech as first responders to customer escalations.
		\item Improving monitoring infrastructure for Bidding alerts in collaboration with the Smart Alerting team. Establishing semi-automated proactive threshold adjustment processes. Creating debugging tools for the first responders on Bidding alerts.
		\item Building experiment analysis system for pMax in collaboration with channel bidder teams. Working on automated analytics solutions for quick response on pMax outages.
		\item Working on Metabidding solutions to facilitate migrations of different campaign types to pMax, collaborating with GDA, Search and Shopping teams.
		\item Leading effort to optimize bidding behavior for budget restricted customers, which \textbf{increased spend and bidding quality by +3\%}.
		\item Designing and implementing cold start improvements for one of the core bidding algorithms, which serves on \textbf{45\% of Shopping Ads traffic}, resulting in \textbf{30\% of small customers starting to have traffic}.
	\end{itemize}
}
\cventry{May 2018 -- May 2021}{Google Switzerland}{Senior Software engineer}{Shopping Bidding Team}{}{
	\begin{itemize}
		\item Working on cutting edge bidding solutions, which account for a significant chunk of Google Shopping revenue.
		\item Detected several major bugs in autobidding implementation, fixes of which statistically significantly improved the quality of traffic \textbf{(+5\% CVR uplift on enhanced CPC bidding strategy)}.
		\item Implemented new bidding strategies and solutions, which did not exist in Google Shopping before (enhanced CPC for Value, Cooperative bidding, target ROAS improvements).
		\item Used Recurrent Neural Networks for session-based optimizations, \textbf{brought AUC-PR up by 20\%} for a TFX based model.
		\item Improving user trust by implementing bidding strategy statuses logic for Smart Shopping campaigns.
		\item Helped facilitating cross org effort to migrate external ad stack (SA360) to Google Ads infra (Amalgam project), which involved \textbf{communicating directly with over 4 teams}, took ownership of pieces outside of the scope of my team
		\item Occasionally taking and solving complicated Google-wide infra issues. Examples include fixing bug in TensorFlow matrices operations which \textbf{brought +8.8\% AUC-PR and -1\% logloss} to one of the core models used in ads auction for all the Shopping traffic. Another example is fixing googlesql python client when it was breaking in multithreading context under python3, which facilitated company-wide python3 migration.
	\end{itemize}
}
\cventry{Jun 2015 -- March 2018}{Yandex}{Eng Manager}{Advertising Services, Autobidding Team}{}{
	\begin{itemize}
		\item Yandex is a search engine, Google's main competitor on the Russian market. Yandex has achieved more than 50\% share in search engine usage (compared with Google, Bing, and other search engines) among Russian audience due to better interpretability capabilities for Russian language. It's one of the largest tech companies in Russia.
		\item I was managing a team of 5 people.
		\item Being responsible for building new and improving existing Autobidding solutions for Yandex Direct.
		\item Working with PMs to on customer feedback about Autobidding products, categorizing issues and coming up with roadmap for improvements.
		\item Collaborating with other Ads backend teams whenever a bidding product required changes in API, data collection and processing, real time logic modifications, etc.
	\end{itemize}
}
\cventry{Jul 2012 -- Jun 2015}{Yandex}{Software Engineer}{Advertising Services, Fraud Detection}{}{
	\begin{itemize}
		\item Performing complex data analysis over users data and Yandex Ads Partner Network's data to determine fraudulent activities and automated ways to catch them
		\item Developing automatic Alerting system for Yandex Ads. I've joined Yandex early when there were no established systems for data monitoring in Ads. I was designing the system from scratch, collected requirements from multiple Ads teams, and have worked with a group of 3-4 engineers to build it out.
		\item Designing and implementing Anomalies Analyzer system. The system was capable of performing deep automated analysis of outages and fraud complaints by slicing the logs data using a predefined set of slices and a set of data analysis algorithms. The system was designed in a modular, extensible way, it was possible to plug in multiple automated analytics procedures. I have designed some of the automated analytics modules myself, while designing some others according to data scientists feedback.
		\item I integrated the Anomalies Analyzer system with our Alerting system. Anomalies in advertisement traffic were automatically investigated by automation before a human person takes a look. In most cases the automated system was capable of producing good quality results, e.g. identifying ip addresses from fraud complaints and auto-adding them to ban list as part of custom post-processing.
		\item Developing the backend of an eventually consistent consensus system for Ads backend. The system worked similar to RAFT, but was designed for offline usage, on top of MySQL, for executing MySQL queries synchronously with eventual consistency guarantees..
		\item Developing word2vec recommendation models for Ads on Yandex Partner Network.
	\end{itemize}
}

\section{Skills}

\cventry{Programming languages}{Proficiency level: C++, Python, Perl, R, JavaScript (with JQuery), html, SQL \newline Intermediate level: Scala}{}{}{}{}
\cventry{ML tools}{Pandas, NumPy, SciPy, scikit-learn, matplotlib, Apache Spark, Vowpal Wabbit, word2vec}{}{}{}{}
\cventry{Other tools}{linux, git, svn, deb packages, gdb, pdb, vim, flask, nginx, mongodb, mac os}{}{}{}{}
\cventry{Languages}{Russian (native), English (proficiency)}{}{}{}{}

\section{Publications}
\cventry{May 2015}{Journal of Mathematical Sciences}{Extension of Endomorphisms of the Subsemigroup $GE_2+(R)$ to Endomorphisms of $GE_2+(R[x])$, Where R is a Partially-Ordered Commutative Ring Without Zero Divisors}{Volume 206, Issue 6, pp 711-733 \myurl{https://link.springer.com/article/10.1007\%2Fs10958-015-2348-y}}{}{}
\cventry{Sep 2014}{Journal of Mathematical Sciences}{Endomorphisms of the Semigroup $G_2(R)$ Over Partially Ordered Commutative Rings Without Zero Divisors and with 1/2}{Volume 201, Issue 4, pp 534-551 \myurl{https://link.springer.com/article/10.1007/s10958-014-2010-0} }{}{}
\cventry{2013}{Fundamental and Applicable Mathematics}{Extension of endomorphisms of the semigroup ${\mathrm{GE}}_2+(R)$ to endomorphisms of ${\mathrm{GE}}_2+(R[x])$ for the lattice-ordered commutative ring $R$ with a unit and without zero divisors.}{no. 4, pp 155-184 \myurl{http://www.ams.org/mathscinet-getitem?mr=3431839}}{}{}

\section{Projects}
\cventry{2017 -- now}{RAST -- a highly distributed database}{}{\myurl{https://github.com/olegtsts/RAST}, \myurl{https://github.com/olegtsts/parallel_programming}}{}{
	\begin{itemize}
		\item Project is in the designing state.
		\item Involves methods of constructing fault-tolerant service.
		\item Includes:
		\begin{itemize}
			\item high-level multithreading technique
			\item exception-safe design techniques
			\item safe memory release techniques
		\end{itemize}
		\item See the example of code here: \myurl{https://github.com/olegtsts/parallel_programming/blob/master/lock-free-queue-with-olegts-ref-counting.cpp}
	\end{itemize}
}
\cventry{2015 -- 2016}{Graph Analyzer Utility}{Mentoring project in Higher School of Economics}{\myurl{https://github.com/ilyshnikova/graph-analyzer}}{}{
	\begin{itemize}
		\item Server-based program, which accepts time series points and detects anomalies on data series.
		\item Works online with complexity $\underline{O}(1)$ on point submission.
		\item Includes human html interface for controlling algorithms, which work on time series.
		\item Source code: \myurl{https://github.com/ilyshnikova/graph-analyzer}
	\end{itemize}
}
Git repository: \myurl{https://github.com/olegtsts/}

\section{Olympiads}
\cventry{2011}{International Mathematics Competition for University Students}{}{Prize winner (third prize)}{}{}
\cventry{2011}{Students Olympiad of Higher Algebra}{}{Prize winner (second prize)}{}{}
\cventry{2010}{Students Olympiad of Higher Algebra}{}{Prize winner (third prize)}{}{}
\cventry{Apr 2009}{All-Russian Olympiad of Mathematics}{}{Prize winner (second prize)}{}{}
\cventry{Apr 2009}{Moscow State Olympiad of Mathematics}{}{Prize winner (first prize)}{}{}
\cventry{Jul 2008, 2007, 2006}{Geometrical Olympiad in Honor of I.F.Sharygin}{}{Prize winner (first prize), three times}{}{}
\cventry{Apr 2008}{All-Russian Olympiad of Mathematics}{}{Prize winner (first prize)}{}{}
\cventry{Mar 2008}{Moscow State Olympiad of Mathematics}{}{Prize winner (third prize)}{}{}
\cventry{Apr 2007}{All-Russian Olympiad of Mathematics}{}{Prize winner (second prize)}{}{}
\cventry{Apr 2006}{All-Russian Olympiad of Mathematics}{}{Prize winner (third prize)}{}{}
\cventry{Mar 2006}{Moscow State Olympiad of Mathematics}{}{Prize winner (second prize)}{}{}

\end{document}
